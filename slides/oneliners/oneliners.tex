\documentclass[ignorenonframetext,]{beamer}
\setbeamertemplate{caption}[numbered]
\setbeamertemplate{caption label separator}{: }
\setbeamercolor{caption name}{fg=normal text.fg}
\beamertemplatenavigationsymbolsempty
\usepackage{lmodern}
\usepackage{amssymb,amsmath}
\usepackage{ifxetex,ifluatex}
\usepackage{fixltx2e} % provides \textsubscript
\ifnum 0\ifxetex 1\fi\ifluatex 1\fi=0 % if pdftex
  \usepackage[T1]{fontenc}
  \usepackage[utf8]{inputenc}
\else % if luatex or xelatex
  \ifxetex
    \usepackage{mathspec}
  \else
    \usepackage{fontspec}
  \fi
  \defaultfontfeatures{Ligatures=TeX,Scale=MatchLowercase}
    \setmainfont[]{Avenir}
    \setmonofont[Mapping=tex-ansi]{Menlo}
\fi
\usefonttheme{serif} % use mainfont rather than sansfont for slide text
% use upquote if available, for straight quotes in verbatim environments
\IfFileExists{upquote.sty}{\usepackage{upquote}}{}
% use microtype if available
\IfFileExists{microtype.sty}{%
\usepackage{microtype}
\UseMicrotypeSet[protrusion]{basicmath} % disable protrusion for tt fonts
}{}
\newif\ifbibliography
\usepackage{color}
\usepackage{fancyvrb}
\newcommand{\VerbBar}{|}
\newcommand{\VERB}{\Verb[commandchars=\\\{\}]}
\DefineVerbatimEnvironment{Highlighting}{Verbatim}{commandchars=\\\{\}}
% Add ',fontsize=\small' for more characters per line
\newenvironment{Shaded}{}{}
\newcommand{\KeywordTok}[1]{\textcolor[rgb]{0.00,0.44,0.13}{\textbf{{#1}}}}
\newcommand{\DataTypeTok}[1]{\textcolor[rgb]{0.56,0.13,0.00}{{#1}}}
\newcommand{\DecValTok}[1]{\textcolor[rgb]{0.25,0.63,0.44}{{#1}}}
\newcommand{\BaseNTok}[1]{\textcolor[rgb]{0.25,0.63,0.44}{{#1}}}
\newcommand{\FloatTok}[1]{\textcolor[rgb]{0.25,0.63,0.44}{{#1}}}
\newcommand{\ConstantTok}[1]{\textcolor[rgb]{0.53,0.00,0.00}{{#1}}}
\newcommand{\CharTok}[1]{\textcolor[rgb]{0.25,0.44,0.63}{{#1}}}
\newcommand{\SpecialCharTok}[1]{\textcolor[rgb]{0.25,0.44,0.63}{{#1}}}
\newcommand{\StringTok}[1]{\textcolor[rgb]{0.25,0.44,0.63}{{#1}}}
\newcommand{\VerbatimStringTok}[1]{\textcolor[rgb]{0.25,0.44,0.63}{{#1}}}
\newcommand{\SpecialStringTok}[1]{\textcolor[rgb]{0.73,0.40,0.53}{{#1}}}
\newcommand{\ImportTok}[1]{{#1}}
\newcommand{\CommentTok}[1]{\textcolor[rgb]{0.38,0.63,0.69}{\textit{{#1}}}}
\newcommand{\DocumentationTok}[1]{\textcolor[rgb]{0.73,0.13,0.13}{\textit{{#1}}}}
\newcommand{\AnnotationTok}[1]{\textcolor[rgb]{0.38,0.63,0.69}{\textbf{\textit{{#1}}}}}
\newcommand{\CommentVarTok}[1]{\textcolor[rgb]{0.38,0.63,0.69}{\textbf{\textit{{#1}}}}}
\newcommand{\OtherTok}[1]{\textcolor[rgb]{0.00,0.44,0.13}{{#1}}}
\newcommand{\FunctionTok}[1]{\textcolor[rgb]{0.02,0.16,0.49}{{#1}}}
\newcommand{\VariableTok}[1]{\textcolor[rgb]{0.10,0.09,0.49}{{#1}}}
\newcommand{\ControlFlowTok}[1]{\textcolor[rgb]{0.00,0.44,0.13}{\textbf{{#1}}}}
\newcommand{\OperatorTok}[1]{\textcolor[rgb]{0.40,0.40,0.40}{{#1}}}
\newcommand{\BuiltInTok}[1]{{#1}}
\newcommand{\ExtensionTok}[1]{{#1}}
\newcommand{\PreprocessorTok}[1]{\textcolor[rgb]{0.74,0.48,0.00}{{#1}}}
\newcommand{\AttributeTok}[1]{\textcolor[rgb]{0.49,0.56,0.16}{{#1}}}
\newcommand{\RegionMarkerTok}[1]{{#1}}
\newcommand{\InformationTok}[1]{\textcolor[rgb]{0.38,0.63,0.69}{\textbf{\textit{{#1}}}}}
\newcommand{\WarningTok}[1]{\textcolor[rgb]{0.38,0.63,0.69}{\textbf{\textit{{#1}}}}}
\newcommand{\AlertTok}[1]{\textcolor[rgb]{1.00,0.00,0.00}{\textbf{{#1}}}}
\newcommand{\ErrorTok}[1]{\textcolor[rgb]{1.00,0.00,0.00}{\textbf{{#1}}}}
\newcommand{\NormalTok}[1]{{#1}}
\usepackage{longtable,booktabs}
\usepackage{caption}
% These lines are needed to make table captions work with longtable:
\makeatletter
\def\fnum@table{\tablename~\thetable}
\makeatother

% Prevent slide breaks in the middle of a paragraph:
\widowpenalties 1 10000
\raggedbottom

\AtBeginPart{
  \let\insertpartnumber\relax
  \let\partname\relax
  \frame{\partpage}
}
\AtBeginSection{
  \ifbibliography
  \else
    \let\insertsectionnumber\relax
    \let\sectionname\relax
    \frame{\sectionpage}
  \fi
}
\AtBeginSubsection{
  \let\insertsubsectionnumber\relax
  \let\subsectionname\relax
  \frame{\subsectionpage}
}

\setlength{\emergencystretch}{3em}  % prevent overfull lines
\providecommand{\tightlist}{%
  \setlength{\itemsep}{0pt}\setlength{\parskip}{0pt}}
\setcounter{secnumdepth}{0}
\usepackage{pandoc-solarized}
\hypersetup{colorlinks=true, linkcolor=sblue, urlcolor=sblue, breaklinks=true}

\setbeamercolor{normal text}{fg=sbase02,bg=sbase3}
\setbeamercolor{structure}{fg=sblue}

\setbeamerfont{title}{size=\Huge}



\title{Oneliners}
\author{Jake Zimmerman}
\date{\today}

\begin{document}
\frame{\titlepage}

\section{Review}\label{review}

\begin{frame}[fragile]{We can define variables in bash}

\begin{Shaded}
\begin{Highlighting}[]
\CommentTok{# set my_variable to the string "hello"}
\CommentTok{# (no spaces around the '=')}
\NormalTok{$ }\OtherTok{my_variable=}\StringTok{"hello"}

\CommentTok{# get the value of my_variable and print it}
\NormalTok{$ }\KeywordTok{echo} \OtherTok{$my_variable}
\NormalTok{hello}

\CommentTok{# print another_var surrounded by other text}
\NormalTok{$ }\OtherTok{another_var=}\StringTok{"some string"}
\NormalTok{$ }\KeywordTok{echo} \NormalTok{lone}\OtherTok{$\{another_var\}}\NormalTok{s}
\NormalTok{lonesome }\KeywordTok{strings}

\CommentTok{# Sometimes using \{...\} is important:}
\NormalTok{$ }\KeywordTok{echo} \NormalTok{lone}\OtherTok{$another_vars}
\NormalTok{lone}
\end{Highlighting}
\end{Shaded}

\end{frame}

\begin{frame}{Quotes can be used to group arguments}

\begin{itemize}
\tightlist
\item
  No quotes

  \begin{itemize}
  \tightlist
  \item
    spaces separate arguments
  \end{itemize}
\item
  Single or double quotes

  \begin{itemize}
  \tightlist
  \item
    entire quoted argument is one argument
  \item
    spaces inside don't break it up
  \end{itemize}
\end{itemize}

\end{frame}

\begin{frame}[fragile]{Quotes are optional sometimes}

\begin{Shaded}
\begin{Highlighting}[]
\CommentTok{# Unquoted strings are still strings}
\NormalTok{$ }\KeywordTok{echo} \NormalTok{hello}
\NormalTok{hello}

\CommentTok{# Quoted strings are strings}
\NormalTok{$ }\KeywordTok{echo} \StringTok{'hello'}
\NormalTok{hello}
\end{Highlighting}
\end{Shaded}

\end{frame}

\begin{frame}[fragile]{Quotes aren't optional with special characters}

\begin{Shaded}
\begin{Highlighting}[]
\CommentTok{# Unquoted special characters are shell expanded}
\NormalTok{$ }\KeywordTok{echo} \NormalTok{*}
\NormalTok{file1.txt folder}

\CommentTok{# Quoted special characters aren't expanded}
\NormalTok{$ }\KeywordTok{echo} \StringTok{"*"}
\NormalTok{*}

\CommentTok{# ...unless the special character is a '$'}
\NormalTok{$ }\KeywordTok{echo} \StringTok{"}\OtherTok{$my_variable}\StringTok{"}
\NormalTok{hello}

\CommentTok{# ...in which case we can use single quotes}
\NormalTok{$ }\KeywordTok{echo} \StringTok{'$my_variable'}
\OtherTok{$my_variable}
\end{Highlighting}
\end{Shaded}

\end{frame}

\begin{frame}[fragile]{We don't \emph{have} to quote; we can escape}

\begin{Shaded}
\begin{Highlighting}[]
\CommentTok{# We can use '\textbackslash{}' to escape special characters}
\NormalTok{$ }\KeywordTok{echo} \DataTypeTok{\textbackslash{}*}
\NormalTok{*}

\CommentTok{# Escaping and quoting compound on each other}
\CommentTok{# (Sometimes this is what we want)}
\NormalTok{$ }\KeywordTok{echo} \StringTok{"\textbackslash{}*"}
\DataTypeTok{\textbackslash{}*}
\end{Highlighting}
\end{Shaded}

\end{frame}

\begin{frame}[fragile]{Understand the difference between globs and
regex}

\begin{Shaded}
\begin{Highlighting}[]
\KeywordTok{grep} \NormalTok{homework_problem}\KeywordTok{(.}\NormalTok{*}\KeywordTok{)} \NormalTok{homework.py}
\CommentTok{#                      ^}
\CommentTok{# This is a special shell character!}
\CommentTok{# We need to quote or escape it.}

\KeywordTok{grep} \StringTok{'homework_problem(.*)'} \NormalTok{homework.py}
\CommentTok{#                       ^}
\CommentTok{# It's quoted now--we're good.}
\end{Highlighting}
\end{Shaded}

\end{frame}

\section{Input \& Output}\label{input-output}

\begin{frame}[fragile]{\texttt{stdin}, \texttt{stdout}, \texttt{stderr}}

\begin{itemize}
\tightlist
\item
  Each process\ldots{}

  \begin{itemize}
  \tightlist
  \item
    can listen for text input on \texttt{stdin} (standard input)
  \item
    can output ``normal'' text on \texttt{stdout} (standard output)
  \item
    can output ``error'' text on \texttt{stderr} (standard error)
  \end{itemize}
\end{itemize}

\end{frame}

\begin{frame}[fragile]{Redirection}

\begin{itemize}
\tightlist
\item
  Normally \texttt{stdin} is the keyboard, and \texttt{stdout} \&
  \texttt{stderr} are the terminal
\item
  We can change this

  \begin{itemize}
  \tightlist
  \item
    We Have The Technology™
  \end{itemize}
\end{itemize}

\begin{longtable}[]{@{}ll@{}}
\toprule
Syntax & Meaning\tabularnewline
\midrule
\endhead
\texttt{command\ \textless{}\ file.txt} & \texttt{stdin} from
\texttt{file.txt}\tabularnewline
\texttt{command\ \textgreater{}\ file.txt} & \texttt{stdout} to
\texttt{file.txt} (overwrite)\tabularnewline
\texttt{command\ \textgreater{}\textgreater{}\ file.txt} &
\texttt{stdout} to \texttt{file.txt} (append)\tabularnewline
\texttt{command\ 2\textgreater{}\ file.txt} & \texttt{stderr} to
\texttt{file.txt} (overwrite)\tabularnewline
\texttt{command\ 2\textgreater{}\textgreater{}\ file.txt} &
\texttt{stderr} to \texttt{file.txt} (append)\tabularnewline
\bottomrule
\end{longtable}

\end{frame}

\begin{frame}[fragile]{\texttt{/dev/null} is a ``black hole'' file}

\begin{itemize}
\tightlist
\item
  Anything sent to \texttt{/dev/null} is thrown away
\item
  Anything read from \texttt{/dev/null} is empty
\end{itemize}

\end{frame}

\begin{frame}[fragile]{Pipes send \texttt{stdout} of one command to
\texttt{stdin} of another}

\begin{Shaded}
\begin{Highlighting}[]
\CommentTok{# Disclaimer: this is a toy example.}

\CommentTok{# We normally run grep like this:}
\KeywordTok{grep} \NormalTok{TODO *}

\CommentTok{# But if we give grep no arguments, it}
\CommentTok{# will search on stdin. So we can do}
\CommentTok{# this equivalent command}
\KeywordTok{cat} \NormalTok{* }\KeywordTok{|} \KeywordTok{grep} \NormalTok{TODO}
\end{Highlighting}
\end{Shaded}

\end{frame}

\section{Oneliners}\label{oneliners}

\begin{frame}[fragile]{Oneliners are chains of pipes}

\begin{itemize}
\tightlist
\item
  We start with some sort of data
\item
  Then we filter it down
\end{itemize}

\begin{Shaded}
\begin{Highlighting}[]
\CommentTok{# Example}
\NormalTok{$ }\KeywordTok{du} \NormalTok{-h d1 }\KeywordTok{|} \KeywordTok{sort} \NormalTok{-hr}
\CommentTok{# ^          ^}
\CommentTok{# |          |}
\CommentTok{# |          `- And feed it to this command (filter)}
\CommentTok{# `- Take the stdout of this command (initial data)}
\end{Highlighting}
\end{Shaded}

\end{frame}

\begin{frame}[fragile]{Useful commands}

\begin{itemize}
\tightlist
\item
  Old:

  \begin{itemize}
  \tightlist
  \item
    \texttt{sed}
  \item
    \texttt{grep}
  \end{itemize}
\item
  New:

  \begin{itemize}
  \tightlist
  \item
    \texttt{find}

    \begin{itemize}
    \tightlist
    \item
      \texttt{-name}
    \item
      \texttt{-regex}
    \end{itemize}
  \item
    \texttt{curl}
  \item
    \texttt{xargs}
  \end{itemize}
\end{itemize}

\end{frame}

\begin{frame}[fragile]{Examples}

\begin{Shaded}
\begin{Highlighting}[]
\CommentTok{# Open all PDF files not named written.pdf}
\KeywordTok{find} \KeywordTok{.} \NormalTok{-name }\StringTok{"*pdf"} \KeywordTok{\textbackslash{}}
  \KeywordTok{|} \KeywordTok{grep} \NormalTok{-v }\StringTok{"written.pdf"} \KeywordTok{\textbackslash{}}
  \KeywordTok{|} \KeywordTok{xargs} \NormalTok{open}

\CommentTok{# Get 100 random lowercase dictionary words}
\NormalTok{shuf /usr/share/dict/words }\KeywordTok{\textbackslash{}}
  \KeywordTok{|} \KeywordTok{head} \NormalTok{-n 100 }\KeywordTok{\textbackslash{}}
  \KeywordTok{|} \KeywordTok{tr} \StringTok{'[A-Z]'} \StringTok{'[a-z]'} \KeywordTok{\textbackslash{}}
  \KeywordTok{|} \KeywordTok{sort}

\CommentTok{# Count how many times it says "Vim" on a page}
\NormalTok{curl https://jez.io }\KeywordTok{\textbackslash{}}
  \KeywordTok{|} \KeywordTok{grep} \NormalTok{--only-matching Vim }\KeywordTok{\textbackslash{}}
  \KeywordTok{|} \KeywordTok{wc} \NormalTok{-l}
\end{Highlighting}
\end{Shaded}

\end{frame}

\section{Recap}\label{recap}

\begin{frame}{Tips for Writing Oneliners}

\begin{itemize}
\tightlist
\item
  Construct oneliners iteratively!

  \begin{itemize}
  \tightlist
  \item
    Try the first command, see what it outputs
  \item
    Try the first two commands, see what they output
  \item
    \ldots{}
  \end{itemize}
\item
  Many tools do the same thing

  \begin{itemize}
  \tightlist
  \item
    Choose what you're familiar with
  \end{itemize}
\item
  Some tools are subtly different

  \begin{itemize}
  \tightlist
  \item
    For example, not all commands have the same regex syntax
  \end{itemize}
\end{itemize}

\end{frame}

\begin{frame}[fragile]{More resources}

\begin{itemize}
\tightlist
\item
  Google is great for finding the \emph{filtering} commands

  \begin{itemize}
  \tightlist
  \item
    ``Strings that don't match\ldots{}''
  \item
    ``Sum a list of numbers''
  \item
    ``Replace character with newline''
  \end{itemize}
\item
  \ldots{} but don't just run what people tell you!

  \begin{itemize}
  \tightlist
  \item
    \texttt{man} pages
  \item
    \url{http://explainshell.com}
  \end{itemize}
\end{itemize}

\end{frame}

\end{document}
